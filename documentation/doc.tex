\documentclass{article}
\usepackage[utf8]{inputenc}
\usepackage[OT4]{fontenc}
\usepackage{polski}
\usepackage{lmodern}
\usepackage[colorlinks=true, allcolors=blue]{hyperref}
\usepackage[a4paper]{geometry}
\usepackage{graphicx}

% # TODO: Add a stylised code block for use in code snippets and function names
% # TODO: Stylise the page with 'geometry' package
% # TODO: Make a title page for the document
% # TODO: Format the style of the document

\graphicspath{ {./img/} }

\input{tex/titlepage.tex}

\setDepartment{Elektronika I}
\setGroup{czwartek 14:40}
\setFaculty{Elektronika I}
\setReviewer{Rafał Frączek}
\setSupervisor{Rafał Frączek}
\setMajor{Dupa}

\title{Connect Four}
\author{Kacper Filipek}
\date{\today}

\begin{document}

\fontfamily{cmss}
\selectfont

\maketitle

\newpage

\section{Opis projektu}

Projekt jest realizcją gry w "Czwórki" (ang. "Connect Four") w C++, z interfejsem TUI (t.j. interfejs używający znaków specjalnych i kolorów do rysowania interfejsu użytkownika w konsoli). 

\section{Project description}

The project is a realisation of the game "Connect Four" made in C++ with TUI interface (an interface using special characters for drawing UI in the terminal)

\section{Instrukcja użytkownika}

\section{Kompilacja}

Program został napisany na systemy operacyjne z rodziny Linux, 
chociaż powinien on działać na Windowsie, ponieważ kod nie 
używa żadnych zależnych od platformy plików nagłówkowych. 
Program używa  systemu CMake do budowania projektu. 
Można skompilować go na dwa sposoby:
    \begin{itemize}
         
    \item Sposób 1:
        \begin{enumerate}
            \item Wejść do folderu build/
            \item Wykonać polecenie cmake ..
        \end{enumerate}

    \item Sposób 2:
        \begin{enumerate}
            \item Z katalogu głównego wywołać skrypt ./bld.sh. Ze względu na fakt, że skrypt wywołuje program make, może on nie działać na Windowsie.
        \end{enumerate}

    \end{itemize}

Po zbudowaniu powienien się plik wykonywalny o ./build/connect-four. 
Z uwagi na fakt, że ścieżki do zasobów są wpisane w programie relatywnie 
do głównego katalogu, to program wykonywalny powinien z niego wywoływany.

\section{Pliki źródłowe}

W tym punkcie należy opisać wszystkie pliki źródłowe (.cpp, .h) w projekcie. Należy podać nazwę każdego pliku oraz informację o tym co się w nim znajduje. Na przykład:
Projekt składa się z następujących plików źródłowych:
    \begin{itemize}
    \item board.h, board.cpp – deklaracja oraz implementacja klasy Board,
    \item game.h, game.cpp – deklaracja oraz implementacja klasy Game,
    \item menu.h, menu.cpp – deklaracja oraz implementacja klasy Menu.
    \item extras.h, extras.cpp – deklaracja oraz implementacja funkcji pomocniczych.
    \item color.h – definicje procesora nazw kolorów do użycia w funkcjach biblioteki ncurses.
    \item main.cpp – główny plik z implementacją funkcji main.
    \end{itemize}

\section{Zależności}

W projekcie wykorzystano następujące dodatkowe biblioteki:
    \begin{itemize}
    \item ncurses – biblioteka do interakcji z emulatorem terminala,  
    pozwala  tworzyć zaawansowane interfejsy konsolowe:\\
    \url{https://invisible-island.net/ncurses/}
    \end{itemize}
\section{Opis klas}

W projekcie utworzono następujące klasy:
\begin{itemize}
    \item Menu
        \begin{itemize}
            \item Menu()
            \item void Draw()
            \item void Update()
            \item void Start()
            \item void Loop()
            \item void key\_handler()
            \item void next\_option()
            \item void prev\_option()
        \end{itemize}

    \item Game
        \begin{itemize}
            \item void Game()
            \item void Start()
            \item void Loop()
        \end{itemize}

    \item Board
        \begin{itemize}
            \item Board()
            \item get\_columns()
            \item get\_rows()
        \end{itemize}
\end{itemize}
     
\section{Zasoby}

W projekcie wykorzystywane są następujące pliki zasobów:
    \begin{itemize}
    \item assets/ – katalog zawierający dodatkowe zasoby do gry.\\
    Struktura katalogu:
        \begin{itemize}
            \item logo1.txt, logo2.txt – pliki zawierające tekstowe logo pojawiające się na ekranie startowym.
        \end{itemize}
    \end{itemize}

\section{Dalszy rozwój i ulepszenia}

\section{Inne}


\end{document}

