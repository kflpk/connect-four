\documentclass{article}
\usepackage[utf8]{inputenc}
\usepackage[OT4]{fontenc}
\usepackage{polski}
% \usepackage{lmodern}
\usepackage[colorlinks=true, allcolors=blue]{hyperref}
\usepackage[a4paper, margin={2.54cm, 1.54cm}]{geometry}
\usepackage{graphicx}
\usepackage{titlesec}
\usepackage{titling}
\usepackage{enumitem}

% # TODO: Add a stylised code block for use in code snippets and function names
% # TODO: Stylise the page with 'geometry' package
% # TODO: Make a title page for the document
% # TODO: Format the style of the document

\titleformat{\section}
{\LARGE \bfseries}
{\thesection.}
{0.5em}
{}[\titlerule]

\titleformat{\subsection}
{\Large \bfseries}
{\thesubsection.}
{0.5em}
{}[]

% \renewcommand{\labelenumii}{\theenumi.\theenumii.}
\setlist[enumerate]{label*=\arabic*.}


\graphicspath{ {./img/} }

\input{tex/titlepage.tex}

\setDepartment{Elektronika I}
\setGroup{czwartek 14:40}
\setFaculty{Elektronika I}
\setReviewer{Rafał Frączek}
\setSupervisor{Rafał Frączek}

\title{Connect Four}
\author{Kacper Filipek}
\date{\today}

\begin{document}

\fontfamily{cmss}
\selectfont

\maketitle

\newpage

\section{Opis projektu}

Projekt jest realizcją gry w "Czwórki" (ang. "Connect Four") w C++, z interfejsem TUI (t.j. interfejs używający znaków 
specjalnych i kolorów do rysowania interfejsu użytkownika w konsoli). 

\section{Project description}

The project is a realisation of the game "Connect Four" made in C++ with TUI interface (an interface using special 
characters for drawing UI in the terminal)

\section{Instrukcja użytkownika}
\subsection*{Menu główne}

Po włączeniu programu wyświetla się menu główne z następującymi opcjami:
\begin{itemize}
    \item Play - rozpoczyna grę w bieżącymi ustawieniami
    \item Load game - umożliwia wczytanie pliku z zapisem gry
    \item Settings - otwiera menu umożliwiające zmianę ustawień gry
    \item Quit game - wychodzi z programu
\end{itemize}

Nawigacja pomiędzy opcjami odbywa się za pomocą strzałek w górę i w dół. Wciśnięcie klawisza Enter spowoduje wykonanie 
akcji związanej z wybranym elementem menu

Wybranie "Load game" wczyta stan gry z pliku z zapisem \texttt{save.bin}. 
W przypadku błędu w odczycie pliku, wyświetli się komunikat powiadamiający o wystąpieniu błędu.

\subsection*{Ustawienia}
W menu "Settings" znajdują się opcje do zmienienia ustawień gry. Wokół liczby przy wybranej opcji pojawią się strzałki. 
Parametry te można zmienić używając strzałek w lewo i w prawo. Nawigacja pomiędzy opcjami odbywa się za pomocą strzałek
w górę i w dół.
\begin{itemize}
    \item Rows - liczba wierszy planszy (domyślnie 5)
    \item Columns - liczba kolumn planszy (domyślnie 7)
    \item Win condition - liczba żetonów w sekwencji potrzebna do wygrania gry (domyślnie 4)
\end{itemize}
W menu ustawień znajdują się również przyciski "Done" i "Cancel". Gdy użytkownik naciśnie Enter mając
wybraną opcje "Done", ustawienia zostaną zapisane i włączy się główne menu. Wybranie "Cancel" wyjdzie do głównego menu
bez zapisywania zmienionych ustawień.

\subsection*{Ekran gry}

Po rozpoczęciu gry, na ekranie wyświetli okno zostanie podzielone na dwie części. Górna część zawiera elementy
interfejsu informujące o aktualnym graczu i wybranej kolumnie, a dolna - planszę do gry
Zmiana wybranej kolumny odbywa się przy pomocy strzałek w lewo i w prawo. Po wciśnięciu klawisza Enter, do wybranej
kolumny zostanie wrzucony żetonu koloru odpowiadającego aktualnemu graczowi. Po wrzuceniu żetonu następuje tura
kolejnego gracza. Gracz który jako pierwszy ułoży odpowiednią liczbę żetonów w prostej linii lub po przekątnej wygrywa.

Jeśli program zostanie zamknięty w czasie gry lub podczas wyświetlania menu pauzy poprzez wysłanie sygnału 
\texttt{SIGTERM}, aktualny stan gry zostanie zapisany do pliku \texttt{.autosave.bin}. 
Jeśli program zostanie otwarty ponownie, sprawdzi czy plik o tej nazwie istnieje
w aktualnym folderze. Jeśli plik z autozapisem istnieje, zostanie on wczytany bezpośrednio po włączeniu programu,
z pominięciem głownego menu.

Niestety nie jest możliwy automatyczny zapis gry przy zamknięciu okna, ponieważ zamknięty terminal wysyła wszystkim
procesom w nim otwartym sygnał \texttt{SIGKILL}, którego obsłużenie nie jest możliwe.

\subsection*{Menu pauzy}

Wciśnięcie klawisza \texttt{p} podczas gry spowoduje pauzę w grze i wyświetlenie się menu z następującymi opcjami:
\begin{itemize}
    \item Resume - wznawia przebieg gry (alternatywnie można ponownie wcisnąć \texttt{p})
    \item Save game - zapisuje stan gry do pliku \texttt{save.bin}
    \item Main menu - kończy grę i przechodzi do głównego menu bez zapisu
\end{itemize}

\section{Kompilacja}

Program został napisany na systemy operacyjne z rodziny Linux i nie jest możliwe skompilowanie go na system Windows,
ponieważ on Unixowegy systemych sygnałów \texttt{SIGTERM} i \texttt{SIGWINCH}. Prawdopodobnie jest możliwa kompilacja
programu na systemy z rodziny BSD, jednak program nie był na nich testowany.

Program można skompilować na dwa sposoby:
    \begin{itemize}
         
    \item Sposób 1:
        \begin{enumerate}
            \item Wejść do folderu  \texttt{build/}
            \item Wykonać polecenie \texttt{cmake .. \&\& make}
        \end{enumerate}

    \item Sposób 2:
        \begin{enumerate}
            \item Z katalogu głównego wywołać skrypt \texttt{./bld.sh}.
        \end{enumerate}

    \end{itemize}

Po zbudowaniu powienien się plik wykonywalny \texttt{./build/connect-four}.
Z uwagi na fakt, że ścieżki do zasobów są wpisane w programie relatywnie 
do głównego katalogu, to program wykonywalny powinien z niego wywoływany.

\section{Pliki źródłowe}

W tym punkcie należy opisać wszystkie pliki źródłowe (\texttt{.cpp}, \texttt{.h}) w projekcie. Należy podać nazwę 
każdego pliku oraz informację o tym co się w nim znajduje. Na przykład:
Projekt składa się z następujących plików źródłowych:
    \begin{itemize}
    \item \texttt{board.h}, \texttt{board.cpp} – deklaracja oraz implementacja klasy Board,
    \item \texttt{game.h}, \texttt{game.cpp} – deklaracja oraz implementacja klasy Game,
    \item \texttt{menu.h}, \texttt{menu.cpp} – deklaracja oraz implementacja klasy Menu.
    \item \texttt{extras.h}, \texttt{extras.cpp} – deklaracja oraz implementacja funkcji pomocniczych.
    \item \texttt{color.h} – definicje procesora nazw kolorów do użycia w funkcjach biblioteki ncurses.
    \item \texttt{main.cpp} – główny plik z implementacją funkcji main.
    \end{itemize}

\section{Zależności}

W projekcie wykorzystano następujące dodatkowe biblioteki:
    \begin{itemize}
    \item ncurses – biblioteka do interakcji z emulatorem terminala,  
    pozwala tworzyć zaawansowane interfejsy konsolowe:\\
    \url{https://invisible-island.net/ncurses/}
    \end{itemize}
\section{Opis klas}

W projekcie utworzono następujące klasy:
\begin{itemize}
    \item \texttt{Menu} - klasa reprezentująca menu główne programu
        \begin{itemize}
            \item \texttt{Menu()}
            \item \texttt{GameParameters Start()} - rozpoczyna pętlę klasy \texttt{Menu} i zwraca strukturę z 
            parametrami gry po jej zakończeniu
            \item \texttt{void update\_dimensions()} - ustawia wymiary elementów w zależności od rozmiaru okna
        \end{itemize}

    \item \texttt{Game} - klasa reprezentująca stan i właściwy przebieg rogrywki
        \begin{itemize}
            \item \texttt{Game()} - domyślny konstruktor klasy \texttt{Game}
            \item \texttt{void Start()} - ustawia parametry i rozpoczyna pętlę gry
            \item \texttt{bool set\_parameters(GameParameters parameters)} - ustawia parametry gry na podstawie podanego
            argumentu i zwraca informację o powodzeniu metody
            \item \texttt{bool save\_game(const std::string\& path)} - zapisuje stan gry do pliku podanego przez 
            \texttt{path} i zwraca informację o powodzeniu metody
            \item \texttt{void update\_dimensions()} - ustawia wymiary elementów w zależności od rozmiaru okna
        \end{itemize}

    \item \texttt{Board} - klasa reprezentująca planszę do gry
        \begin{itemize}
            \item \texttt{Board()} - domyślny konstruktor, ustawiający domyślne parametry
            \item \texttt{Board(uint8\_t board\_rows, uint8\_t board\_columns)} - konstruktor parametryczny ustawia 
            rozmiar planszy na \texttt{board\_rows} $\times$ \texttt{board\_rows}
            \item \texttt{uint16\_t get\_columns()} - zwraca liczbę kolumn planszy
            \item \texttt{uint16\_t get\_rows()} - zwraca liczbę wierszy planszy
            \item \texttt{uint16\_t get\_win\_condition()} - zwraca warunek zwycięstwa
            \item \texttt{void set\_win\_condition()} - ustawia warunek zwycięstwa
            \item \texttt{void set\_dimensions(uint16\_t rows, uint16\_t columns)} - ustawia wymiary planszy
            \item \texttt{uint8\_t check\_victory()} - sprawdza czy któryś z graczy wygrał
            \item \texttt{void clear()} - czyści planszę
            \item \texttt{std::vector<uint8\_t>::iterator operator[] (int index)} - zwraca iterator do początku podanego
            wiersza. Metoda ta umożliwia odnoszenie się do elementów planszy dwoma indeksami 
            np. \texttt{board[row][col]}
        \end{itemize}
\end{itemize}
     
\section{Zasoby}

W projekcie wykorzystywane są następujące pliki zasobów:
    \begin{itemize}
    \item \texttt{assets/} – katalog zawierający dodatkowe zasoby do gry.\\
    Struktura katalogu:
        \begin{itemize}
            \item \texttt{logo1.txt}, \texttt{logo2.txt} – pliki zawierające tekstowe logo pojawiające się na ekranie 
            startowym.
        \end{itemize}
    \end{itemize}

\section{Dalszy rozwój i ulepszenia}

\begin{itemize}
    \item Mimo, że gra oryginalnie gra została zaprojektowana dla dwóch graczy, to nie ma żadnych powodów dla których
    w grę nie mogłaby grać większa liczba graczy. Zapis gry już przechowuje liczbę graczy, więc implementacja
    funkcji wielu graczy sprowadza się do niewielkich zmian w klasie \texttt{Game} i \texttt{Board} oraz dodanie
    opcji do menu ustawień.
    \item Aktualnie gra nie posiada żadnej możliwości gry dla jednego gracza. Możliwa jest implementacja sztucznej
    inteligencji z różnymi poziomami trudności.
    \item Możliwa jest implementacja logowania ruchów i wyników graczy.
    \item Przy większym nakładzie pracy można zaimplementować funkcjonalność multi-player poprzez protokoły sieciowe,
    jednak wymagałoby to znacznej przebudowy programu.
    \item Gra w obecnym stanie działa w terminalu. Z tego powodu nie jest możliwa obsługa sytuacji w której gracz
    zamyka okno terminala poprzez zastosowanie sygnałów. Problem ten mógłby zostać rozwiązany poprzez napisanie
    faktycznego interfejsu okienkowego (np. w Qt lub GTK), dzięki czemu można by obsługiwać zamknięcie okna.
    \item Alternatywnym rozwiązaniem tego problemu jest zapisywanie gry po każdym ruchu do tymczasowego pliku z zapisem.
    Jeśli program zakończyłby swoje działanie bez problemu, plik mógłby być usunięty. W przypadku wysłania sygnału 
    \texttt{SIGTERM} lub \texttt{SIGKILL}, plik z zapisem pozostawałby w folderze.
\end{itemize}

\section{Inne}

\subsection{Format zapisu}

Stan gry jest zapisywany do pliku w formie binarnej. Poniżej podane są po kolei bajty oznaczające konkretne parametry 
gry oraz ich rozmiary w pliku

\begin{enumerate}
\item Liczba graczy - 1 bajt (unsigned)
\item Aktualny gracz - 1 bajt (unsigned)
\item Wymiary planszy - 4 bajty (unsigned)
  \begin{enumerate}
    \item liczba wierszy w planszy - 2 bajty (unsigned)
    \item liczba kolumn w planszy - 2 bajty (unsigned)
  \end{enumerate}
\item Warunek zwycięstwa - 2 bajty (unsigned)
\item zawartość planszy - każdej komórce odpowiada jeden bajt, więc rozmiar tej sekcji w bajtach, to liczba kolumn 
$\times$ liczba wierszy (unsigned)
\end{enumerate}

Parametry większe niż jeden bajt są zapisywane w formacie little endian (najmniej znaczący bajt na początku).

\end{document}
